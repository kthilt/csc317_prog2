\documentclass[11pt]{article}

%special packages used for symbols, formatting, etc.

%\usepackage{tasks}
\usepackage{courier}
\usepackage{graphicx}
\usepackage{amsmath}
\usepackage{amssymb}
\usepackage[english]{babel}
\usepackage{verbatim}

%Titles
\newcommand{\HWKtitle}[4]{\begin{center}
\textbf{#1\\Program #2\\Title: #3\\Due: #4}
\end{center}\medskip\hrule\bigskip}

\newcommand{\HWKname}[3]{\begin{flushright}
Name: \underline{\textbf{#1}}\\ \underline{\textbf{#2}}\\ \underline{\textbf{#3}}\end{flushright}}

\newcommand{\pdflatexmargins}{
\setlength{\topmargin}{-.5in}
\setlength{\oddsidemargin}{0in}
\setlength{\evensidemargin}{0in}
\setlength{\textwidth}{6.5in}
\setlength{\textheight}{8.75in}}

%Linear Algebra commands
\newcommand{\vce}[4]{\vc{#1} = \langle #2, #3, #4 \rangle}
\newcommand{\vcf}[5]{\vc{#1}(#2) = \langle #3, #4, #5 \rangle}
\newcommand{\vcfcn}[5]{\vc{#1}(#2) = #3\,\vc{\hat{i}} + #4\,\vc{\hat{j}} + #5\,\vc{\hat{k}}}
\newcommand{\veclist}[2]{\vc{#1}_1, \ldots, \vc{#1}_{#2}}
\newcommand{\ip}[2]{\langle #1, #2 \rangle}
\newcommand{\norm}[1]{\left|#1\right|}
\newcommand{\Span}[1]{\textnormal{Span}\left(#1\right)}
\newcommand{\tr}[1]{\textnormal{tr}\left(#1\right)}

%User commands
\newcommand{\vc}[1]{\mathbf{#1}}
\newcommand{\newprob}{\medskip \hrule \medskip}
\newcommand{\rn}[1]{\mathbb{R}^{#1}}
\def\qed{\hspace*{\fill}\rule{1.854mm}{3mm}}

%margins
\pdflatexmargins



\begin{document}

\HWKname{Evan Hammer}{Marcus Berger}{Kevin Hilt}  %%%%%% For those of you who want an obvious name on your LaTeX solution.

\HWKtitle{CSC 317 Program}{\#2}{B-17}{May 1, 2015}


\subsection{Libraries used}
Libraries used in this program are :\\
$\langle iostream \rangle, \langle iomanip \rangle, \langle string \rangle, \langle sstream \rangle, \langle fstream \rangle, and \langle cstdlib \rangle$

\subsection{Algorithms}
\begin{enumerate}
\item Read in the object file and set up the memory array
\item Read in instruction counter and the number of instruction to be executed
\item For each instruction  begin to execute instruction
\begin{enumerate}\item Decode the instruction and determine the address mode and instruction to be executed.
\item Execute the correct instruction \end{enumerate}
\item Check the address mode to get the correct output
\item Output the information for executed instruction 
\item Repeat steps 3-5 for each instruction
\item If an error is encountered 
\end{enumerate}

\subsection{Functions and Program Structure:}
\begin{enumerate}
\item \textbf{Compiling and Usage:}\\
Compile by typing “make b17” in Linux\\
Usage: b17 “object file name”\\
Sample object file:
\verbatiminput{test1.obj}
Sample output file:
\verbatiminput{output1.txt}
\end{enumerate}


\end{document}
 