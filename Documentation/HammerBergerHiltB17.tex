\documentclass[11pt]{article}

%special packages used for symbols, formatting, etc.

%\usepackage{tasks}
\usepackage{courier}
\usepackage{graphicx}
\usepackage{amsmath}
\usepackage{amssymb}
\usepackage[english]{babel}
\usepackage{verbatim}

%Titles
\newcommand{\HWKtitle}[4]{\begin{center}
\textbf{#1\\Program #2\\Title: #3\\Due: #4}
\end{center}\medskip\hrule\bigskip}

\newcommand{\HWKname}[3]{\begin{flushright}
Name: \underline{\textbf{#1}}\\ \underline{\textbf{#2}}\\ \underline{\textbf{#3}}\end{flushright}}

\newcommand{\pdflatexmargins}{
\setlength{\topmargin}{-.5in}
\setlength{\oddsidemargin}{0in}
\setlength{\evensidemargin}{0in}
\setlength{\textwidth}{6.5in}
\setlength{\textheight}{8.75in}}

%Linear Algebra commands
\newcommand{\vce}[4]{\vc{#1} = \langle #2, #3, #4 \rangle}
\newcommand{\vcf}[5]{\vc{#1}(#2) = \langle #3, #4, #5 \rangle}
\newcommand{\vcfcn}[5]{\vc{#1}(#2) = #3\,\vc{\hat{i}} + #4\,\vc{\hat{j}} + #5\,\vc{\hat{k}}}
\newcommand{\veclist}[2]{\vc{#1}_1, \ldots, \vc{#1}_{#2}}
\newcommand{\ip}[2]{\langle #1, #2 \rangle}
\newcommand{\norm}[1]{\left|#1\right|}
\newcommand{\Span}[1]{\textnormal{Span}\left(#1\right)}
\newcommand{\tr}[1]{\textnormal{tr}\left(#1\right)}

%User commands
\newcommand{\vc}[1]{\mathbf{#1}}
\newcommand{\newprob}{\medskip \hrule \medskip}
\newcommand{\rn}[1]{\mathbb{R}^{#1}}
\def\qed{\hspace*{\fill}\rule{1.854mm}{3mm}}

%margins
\pdflatexmargins



\begin{document}

\HWKname{Evan Hammer}{Marcus Berger}{Kevin Hilt}  %%%%%% For those of you who want an obvious name on your LaTeX solution.

\HWKtitle{CSC 317 Program}{\#2}{B-17}{May 1, 2015}

\subsection{Program Description}
The B17 is a 24-bit word-addressable accumulator architecture.  It supports up to 64 instructions, up to four addressing modes, and 212 (4096) words of memory. Main memory consists of 4,096 words, each of which is 24 bits wide. Addresses are 12 bits wide, and range from 000 through FFF. Within the word, bits are numbered from right to left, with bit 0 the least significant bit and bit 23 the most significant bit. The memory interface transfers one word of data at a time. Communication with memory is via two registers, MAR and MDR. Other locations include IC, ABUS, DBUS three registers, AC, ALU and IR. The B17 uses fixed-length instructions. Instructions have the following format an ADDR a twelve-bit operand address (bits 23-12), a six-bit opcode (bits 11-6), and a six-bit addressing mode (bits 5-0). The b17 uses those bits to determine the instruction to be executed from the instruction set. The program continues to execute instructions untill a halt instruction is found or a halting error occurs.

\subsection{Group Assesment}
Kevin Hilt: 35\%\\
Marcus Berger: 35\%\\
Evan Hammer: 30\%\\

\subsection{Libraries used}
Libraries used in this program are :\\
$\langle iostream \rangle, \langle iomanip \rangle, \langle string \rangle, \langle sstream \rangle, \langle fstream \rangle, and \langle cstdlib \rangle$

\subsection{Algorithms}
\begin{enumerate}
\item Read in the object file and set up the memory array
\item Read in instruction counter and the number of instruction to be executed
\item For each instruction  begin to execute instruction
\begin{enumerate}\item Decode the instruction and determine the address mode and instruction to be executed.
\item Execute the correct instruction \end{enumerate}
\item Check the address mode to get the correct output
\item Output the information for executed instruction 
\item Repeat steps 3-5 for each instruction
\item If an error is encountered 
\end{enumerate}

\subsection{Functions and Program Structure:}
int main(int argc, char\* argv[]) - The main function handles file I/O and output formating for the b17 program. Written by Marcus Beger, Evan Hammer, Kevin Hilt using a paired programming approach\\

void get\_address\_mode(int IR, int \&ABUS) - This function pulls out the bits containing the address mode out of the  instruction and sets the ABUS equal to the address mode Written by Marcus Berger\\

void get\_instruction(int IR, string \&instruction) - This function pulls out the bits containing the instruction and sets it equal to a value at an index into an array containing the instruction set. Written by Marcus Beger, Evan Hammer, Kevin Hilt using a paired programming approach\\

void hex\_to\_int(string hex\_string, int\& stored\_int) - This function converts a hex number string to an int variable. Written by Kevin Hilt\\

void read\_memory(int memory[],  char\* filename) - This function pulls out the bits containing the address mode out of the  instruction and sets the ABUS equal to the address mode. Written by Marcus Beger, Evan Hammer, Kevin Hilt using a paired programming approach\\

void match\_instruction(int memory[], int \&MAR, int \&AC, int \&DBUS,  int \&ABUS, int \&IR, int \&IC,  ofstream \&output, string instruction) -This function contains the various code to execute each instruction and the if statements to determine which one to do. Written by Marcus Beger, Evan Hammer, Kevin Hilt using a paired programming approach\\

\subsubsection{Compiling and Usage:}
Compile by typing “make b17” in Linux\\
Usage: b17 “object file name”\\
Sample object file:
\verbatiminput{test1.obj}
Output file will contain memory address, instructions, the instruction executed, the address mode, that contains of the accumulator and the contains of registers x0-x3 for each instruction executed. When a halt instruction or halt error is encountered a correct message is printed and the program terminates. \\
Sample output file:
\verbatiminput{output1.txt}

\subsubsection{Test}
Test 1: Test  ADD, SUB, Jump, Load, and Halt instructions with immediate and direct addressing modes
input:
\verbatiminput{test1.obj}
Output:
\verbatiminput{output1.txt}
Text 2: Test illegal addressing mode
input:
\verbatiminput{test3.txt}
Output:
\verbatiminput{output3.txt}
Test 3: Test clear instruction and unimplemented op code
Input:
\verbatiminput{test4.txt}
Output:
\verbatiminput{output4.txt}

\subsection{What was submitted?}
b17.cpp – contains the main function for the b17 program\\
b17\_functions.cpp – contains the get\_instruction, get\_addressmode, hex\_to\_int, read\_memoy, and match\_instruction functions to prevent needing to repeat code,\\
b17.h – header file containing libraries used and function prototypes for the b17 program\\
makefile – complies the b17 program by ''typing make b17''\\
HammerBergerHiltB17.pdf




\end{document}
 